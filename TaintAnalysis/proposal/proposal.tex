\documentclass[11pt]{article}

\usepackage{graphicx}
\graphicspath{{.}}
\usepackage{amsthm}
\usepackage{listings}
\usepackage{fancyhdr}
\pagestyle{fancy}
\usepackage{indentfirst}
\usepackage{layout}
\usepackage{hanging}
\usepackage{setspace}
\usepackage{mathtools}

\DeclarePairedDelimiter\qb{\lvert}{\rangle}

\def\tit{B433 Paper Proposal}
\def\term{2 March 2020}

\def\auths{Joshua Larkin}

\doublespacing

\lhead{\auths}
\chead{\tit}
\rhead{\thepage}
\cfoot{}

\title{
    \vspace{2in}
    \textmd{\textbf{\tit}}\\
    \normalsize\vspace{0.1in}\small{B433 : Spring 2020 }\\
    \vspace{0.1in}\large{\textit{\auths}}
    \vspace{3in}
}

\date{}

\renewcommand\headrulewidth{0.4pt}
\fancyheadoffset{0.5 cm}

\oddsidemargin 0pt
\evensidemargin 0pt
\topmargin -.3in
\headsep 20pt
%\footskip 20pt
\textheight 8.5in
\textwidth 6.25in

\setlength\topmargin{0pt}
\addtolength\topmargin{-\headheight}
\addtolength\topmargin{-\headsep}
\setlength\oddsidemargin{0pt}
\setlength\textwidth{\paperwidth}
\addtolength\textwidth{-2in}
\setlength\textheight{\paperheight}
\addtolength\textheight{-2in}

\begin{document}

%%%% TITLE PAGE
%\maketitle
%\pagebreak

%%%%% First content page

\section{Introduction}
% When we enter our credit card number into a website, why do we trust that our data is
% safe? Any information we enter in a field of a website is generally trusted because of
% an expectation that the service we are using is responsible. It is convenient,
% albeit a bit naive, to trust that the service provider is taking on the proper
% responsibility to ensure our data is safe. But, I wonder what is actaully being
% done to provide this security. 

A process asks for information, reads it from the 
user, then discards of it when it is done. But how long was the user's information 
available in the program, and where was it available? To get answers, I will research
taint propagation. Taint propagation is the practice of marking points in code
that untrusted sources may have access to. That is, when a user input is asked for,
what methods does the input get used in, where does it get stored in memory, what
code has access to that location, and how long will the input be stored?
Aside from sensitive user information, this has direct relevancy to SQL injection 
attacks as well. This area of study is known as taint propagation.

\section{Approach}
So far I have gathered 4-5 papers on implementations of taint propagation. 
I plan to describe the current state of implementations, and the factors that 
drove researchers to these solutions. For example, FlexiTaint and Iodine are two 
tools I have found papers on. I have heard of implementations for the 
Android device, called TaintDroid, which I would like to cover. I also plan on studying
vulnerabilities found with taint propagation, as well as how programming languages
developers have built taint propagation tools into their language. 

\section{Schedule}
I have outlined the work to be done incrementally over a few weeks.  
One way to read this is that Goal 1 will be done in one week and Goal 
2 in the next. I would like to be ahead of this schedule, though it 
provides a basis for which to make progress. \\

\begin{tabular}{c|c|c}
	2-Week Period & Goal 1 & Goal 2 \\
	\hline 
	28 Feb, 13 Mar & Introduction & Approach \\
	\hline
	13 Mar, 27 Mar & Write the main content & Revise introduction \\
	\hline
	27 Mar, 10 Apr & Revise main content &  Suggestions and Conclusion \\
	\hline
	10 Apr, 24 Apr & Revise Suggestions and Conclusion &  Prepare presentation \\
	\hline
\end{tabular}


\end{document}
